\chapter{Development}
\section{Requirements elicitation}
\subsection{API}
\begin{itemize}
    \item Display all stored exchanges and statistics
    \item Display exchanges with a specific id
    \item Display statistics with a specific id
    \item Display the statistics starting at a specific date
    \item Display the statistics ending at a specific date
    \item Display the statistics with a specific start and end dates
    \item Fetch all the statistics related to a specific exchange
    \item Fetch the statistic related to a particular exchange with a specific id
    \item Fetch all the statistics related to a specific exchange with a specific start date
    \item Fetch all the statistics related to a specific exchange with a particular end date
    \item Fetch a statistic related to a particular exchange with specific start and end dates
\end{itemize}

\subsection{Front-End}
\begin{itemize}
    \item Display for a particular exchange:
    \begin{itemize}
        \item Average fluctuation
        \item Average time in growth
        \item Average time in decline
        \item Difference between highest and latest values
    \end{itemize}
    \item Allow display of information for a period of the last day, week, month and year.
\end{itemize}
\section{Design}
\section{Implementation}
Here is the code for the API:
\subsection{Main Program}
\captionsetup{type=figure}\captionof{figure}{Program.cs}
\subfile{pyg/src/CryptoStats/Program}

\captionsetup{type=figure}\captionof{figure}{Startup.cs}
\subfile{pyg/src/CryptoStats/Startup}

\captionsetup{type=figure}\captionof{figure}{appsettings.json}
\subfile{pyg/src/CryptoStats/appsettings}

\subsection{Models}
\captionsetup{type=figure}\captionof{figure}{Exchange.cs}
\subfile{pyg/src/CryptoStats/Models/Exchange}

\captionsetup{type=figure}\captionof{figure}{Stat.cs}
\subfile{pyg/src/CryptoStats/Models/Stat}

\captionsetup{type=figure}\captionof{figure}{SeedData.cs}
\subfile{pyg/src/CryptoStats/Models/SeedData}

\captionsetup{type=figure}\captionof{figure}{DbContext.cs}
\subfile{pyg/src/CryptoStats/Models/DbContext}

\subsection{Controllers}
\captionsetup{type=figure}\captionof{figure}{ExchangesController.cs}
\subfile{pyg/src/CryptoStats/Controllers/ExchangesController}

\captionsetup{type=figure}\captionof{figure}{ExchangeStatsController.cs}
\subfile{pyg/src/CryptoStats/Controllers/ExchangeStatsController}

\captionsetup{type=figure}\captionof{figure}{StatsController.cs}
\subfile{pyg/src/CryptoStats/Controllers/StatsController}

\subsection{GDAX Client}
\captionsetup{type=figure}\captionof{figure}{GDAXClient.cs}
\subfile{pyg/src/CryptoStats/GDAX/GDAXClient}

\captionsetup{type=figure}\captionof{figure}{ProductsService.cs}
\subfile{pyg/src/CryptoStats/GDAX/Services/Products/ProductsService}

For the GDAX Client, an instance of HttpClient, and JSON deserializers have been used to send HTTP requests to the input API and deserialize the data sent back from the request.

The output API was built using REST technology, and sends JSON responses that the system automatically serializes from the ActionResults that the controllers return.
\section{Testing}
Testing as been done alongside the build of the project, as features were being implemented to make sure that everything was being working as it was supposed to.

Testing as been done on different browsers (Google Chrome, Chromium, Firefox) and also using Sqliteman to make sure that the database was being populated properly.