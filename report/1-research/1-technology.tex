\chapter{Technology - Wider Context}
\section{Considerations on the challenges when building distributed systems}
    When creating distributing systems, different challenges had to be overcome
    by developers in other to develop such systems:
    \begin{itemize}
        \item Inherent complexities: Inter-node communication needs different mechanisms
        than intra-node communication, as well as policies and protocols. Synchronization
        and coordination is more complicated, and problems such as latency, jitter,
        transient failures and overload are also problems to consider.
        \item Accidental complexities: Those happen because of limitations with
        software tools and development techniques. Most of these happen because
        of choices made when the time arose to build the system.
        \item Inadequate methods and techniques: For a long time, software development
        studies and practices have focused on single-process and single-threaded applications,
        which has lead to a lack of expertise on software techniques distributed systems.
        \item Continuous re-invention and re-discovery of core concepts and techniques:
        For a long time, time has been used on solving the same problems over and over
        again, instead of trying to develop techniques that would facilitate the reuse of
        code across different platforms, which lead to less time being available to invest in
        innovation.
    \end{itemize}

    \section{Different Distributed Computing technologies}
    \subsection{Ad hoc Network Programming}
    Ad hoc network programming was one of the first solutions to allow the development
    of distributed computing systems. That technology used Interprocess Communication mechanisms.
    In IPC, there was a use of shared memory, pipes and sockets, which lead to a big coupling between
    the application code and the socket API, as well as paradigms mismatches, since local communication
    techniques use object-oriented techniques and remote communications use function-oriented techniques.

    \subsection{Structured communication}
    This technology was an improvement to ad hoc since it doesn't couple application code
    to low-level IPC mechanisms, offers higher-level communication to distributed systems,
    encapsulates machine-level details and embodies types and communication style closer
    to the application domain. One particular example of this technology are Remote
    Procedure Call platforms. They have the following characteristics:

    \begin{itemize}
        \item Allow distributed applications to cooperate with one another by:
        \begin{itemize}
            \item Invoking functions on each other
            \item Passing parameters along with each invocation
            \item Receiving results from the function that was invocated
        \end{itemize}
    \end{itemize}

    With this technique, however, components are still aware of their peers' remoteness, therefore
    it does not fulfil the following distributed systems requirements:
    \begin{itemize}
        \item Location-independence of components
        \item Flexible component deployment
    \end{itemize}

    \subsection{Middleware}
    Middleware is used as a bridge between the remote system and the application, being a distribution infrastructure
    software. It resides between an application and the operating system, network or database. Different types of
    middleware have appeared over time, which are listed below.

    \subsubsection{Distributed Object Computing}
    Also known as DOC, it represents the confluence of two major information technologies:
    RPC-based distributed computing systems and object-oriented design and programming.

    COBRA 2.x and Java RMI are examples of DOC. They focus on interfaces, which are contracts between clients and
    servers that define a location-independent means for clients to view and access object services provided by a
    server. Such technologies also define communication protocols and object information models to enable
    interoperability.

    Their key limitations are:
    \begin{itemize}
        \item Lack of functional boundaries: This type of middleware treats all interfaces as client/server
        contracts and don't provide standard assembly mechanisms to decouple dependencies among collaborating
        object implementations. This leads to a requirement for the explicit discovery and connection to objects
        that are only dependencies for the actual object that the application is trying to reach.
        \item Lack of software deployment and configuration standards: Results in systems that are harder to
        maintain and software component implementations that are much harder to reuse.
    \end{itemize}

    \subsubsection{Component middleware}
    This technology emerged to address the following limitations of DOC:
    \begin{itemize}
        \item Lack of functional boundaries: Allows a group of cohesive component objects to
        interact with each other through multiple provided and required interfaces and defines standard
        runtime mechanisms needed to execute these component objects in generic application servers.
        \item Lack of standard deployment and configuration mechanisms: Component middleware specifies the
        infrastructure to packages, which makes it easier to customize, assemble and disseminate components
        throughout a distributed system.
    \end{itemize}

    Examples of this technology are Enterprise JavaBeans and COBRA Component Model.

    For this technology, a set of rules and definitions has been applied:
    \begin{itemize}
        \item Component is an implementation entity that exposes a set of named interfaces and connection points
        that components use to communicate with each other.
        \item Containers provide the server runtime environment for component implementations. Contains various
        pre-defined hooks and operations that give components access to strategies and services, such as persistence,
        event notification, transaction, replication, load balancing and security. Each container is also responsible
        for initializing and providing runtime contexts for the managed components.
    \end{itemize}

    Component middleware also automates aspects of various stages in the application development lifecycle like
    component implementation, packaging, assembly and deployment. This approach enables developers
    to create applications more rapidly and robustly than DOC.

    \subsubsection{Publish/Subscribe and Message-Oriented Services}
    RPC, DOC and component middleware are all based on request/response communication. Aspects of this type
    of communication are:
    \begin{itemize}
        \item Synchonous communication: The client waits for a response and stops the rest
        of its process
        \item Designated communication: The client must know the identity
        of the server, which leads to tight coupling between the application code and the API
        \item Point-to-Point communication: Client talks to just one server at a time
    \end{itemize}

    In order to fix the problems that that type of communication created for some systems,
    the following alternatives were created:
    \begin{itemize}
        \item Message-Oriented Middleware: Applied in technologies such as IBM MQ Series, BEA
        MessageQ and TIBCO Rendezvous
        \item Publish/Subscribe middleware: Applied in technologies such as Java Messaging Service
        (JMS), Data Distribution Service (DDS) and WS-NOTIFICATION
    \end{itemize}

    Specifications of Message-Oriented:
    \begin{itemize}
        \item Support for asynchronous communication
        \item Transactional properties
    \end{itemize}

    On top of that, Publish/Subscribe also allows:
    \begin{itemize}
        \item Anonymous communication (loose coupling)
        \item Group communication
    \end{itemize}

    On the Publish/Subscribe paradigm:
    \begin{itemize}
        \item Publishers are the source of events. The may need to describe the type
        of event they generate sometimes
        \item Subscribers are the event sinks; they consume data on topics of interest to them.
        They may need to declare filtering information sometimes
        \item Event channels: Components in the system that propagate events. They can perform
        services such as filtering and routing, Quality of Service enforcement and fault management.
    \end{itemize}

    In these types of systems, events can be represented in many ways, and the interfaces can also be
    generic or specialized.

    \subsubsection{Service-Oriented Architectures and Web Services}
    Service-Oriented Architecture (SOA) is a style of organizing and utilizing distributed capabilities
    that may be controlled by different organizations and owners. It provides a uniform means to offer,
    discover, interact with and use capabilities of loosely coupled and interoperable software services
    to support the requirements of the business processes and application users.

    From the emergence of this technology along with the increasing popularity of the World Wide Web, SOAP
    was created. It is a protocol to exchange XML-based messages over a network, normally using HTTP. SOAP
    spawned a popular new variant of SOA called Web Services. It allows developers to package application
    logic into services whose interfaces are described with the Web Service Description Language (WSDL).
    WSDL-based services are often accessed using higher-level Internet protocols. They can also be used
    to build an Enterprise Service Bus(ESB), which is a distributed computing architecture that simplifies
    interworking between disparate systems.

    Web services have established themselves as the technology of choice for most enterprise business
    applications. They complement earlier successful middleware technologies and provide standard mechanisms
    for interoperability. Examples of Web Services technologies are:
    \begin{itemize}
        \item Microsoft Windows Component Foundation (WCF)
        \item Service Component Architecture (SCA)
    \end{itemize}

    Web Services combine aspects of component-based development and Web technologies. They also provide black-box
    functionality that can be described and reused without concern for how a service is implemented. They are not
    accessed using the object model-specific protocols, but rather using Web protocols and data formats.

    Web service technologies focus on middleware integration \& allow component reuse across an organization's
    entire application set, regardless of the technology implemented for those. In general, web services make it
    relatively easy to reuse and share common logic with such diverse clients as mobile, desktop and web
    applications. The broad reach of web services is possible because they rely on open standards that are
    ubiquitous, interoperable across different computing platforms and independent of the underlying execution
    technologies.

    All web services use HTTP and leverage data-interchange standards like XML and JSON, as well as common media
    types, but differ on the way they may use HTTP:
    \begin{itemize}
        \item As an application protocol to define standard service behaviours
        \item HTTP as a transport mechanism to convey data
    \end{itemize}

    In Web services, there are two prominent design models:
    \begin{itemize}
        \item SOAP
        \begin{itemize}
            \item Language, platform and transport independent
            \item Works well in distributed enterprise environments
            \item Provides significant pre-build extensibility in the
            form of WS* standards
            \item Built-in error handling
            \item XML message-format only
            \item Automation when used with certain language products
            \item Uses WSDL as a description language
        \end{itemize}
        \item REST
        \begin{itemize}
            \item Uses easy to understand standards like swagger and OpenAPI
            Specification 3.0
            \item Can use different types of message formatting like JSON
            \item Fast (no extensive processing required)
            \item Close to other Web technologies in design philosophy
            \item Uses WADL as a description language
        \end{itemize}
    \end{itemize}